\documentclass[11pt,a4paper]{article}
\usepackage[utf8]{inputenc}
\usepackage{graphicx}
\usepackage[left=2.5cm,top=3cm,right=2.5cm,bottom=3cm,bindingoffset=0.5cm]{geometry}
\usepackage{AEDLogica, AEDEspecificacion, AEDTADs}
\usepackage{caratula}

% =======================  TITULO  ======================= %

\titulo{Trabajo práctico}
\subtitulo{Especificación de TADs}

\fecha{\today}

\materia{Algoritmos y Estructuras de Datos}
\grupo{Nombre del grupo} 

\integrante{Hsu, Valentina}{1240/24}{valentina.hsu@gmail.com}
\integrante{Li, Eric}{1094/24}{ericlee05@hotmail.com}
\integrante{Zheng, William}{943/25}{williamzheng73@gmail.com}

\graphicspath{{../static/}}
\newcommand{\Tipo}[1]{\mathsf{#1}}
\newcommand{\norm}[1]{\vert #1\vert}

% =======================  DOCUMENTO  ======================= %

\begin{document}
\maketitle

\section*{Resolución del enunciado}

\begin{tad}{EdR}

% observadores
% \obs{aula}{\seq{\seq{\Z}}}\\
% \obs{solucion}{\seq{\Z}}\\
% \obs{estudiantes}{\dict{LU: {\Z}, examen: \seq{\Z}}}\\

% \begin{proc}{EdR}
% {
%     \In aula: \seq{\seq{\Z}},
%     \In solucion: \seq{\seq{\Z}},
%     \In estudiantes: \Z
% }{
% \Tipo{EjemploDeTAD}
% }
%     \requiere{predicadoUnilinea(caracteres)}
%     \asegura{res.secuenciaDeCaracteres = caracteres}
%     \aseguraLargo{ \norm{res.conjuntoDeTuplas} = \norm{secuenciaDeNombres} \land \\predicadoMultilinea(secuenciaDeNombres, res.conjuntoDeTuplas) }
% \end{proc}

\end{tad}

\end{document}

% \obs{secuenciaDeCaracteres}{\seq{\cha}}
% \obs{conjuntoDeTuplas}{\conj{\tupla{nombre: \seq{\cha}, apellido: \seq{\cha}}}}


% \begin{proc}{crearTAD}
% {
% \In caracteres: \seq{\cha},
% \In secuenciaDeNombres: \seq{\tupla{nombre: \seq{\cha}, apellido: \seq{\cha}}}
% }{
% \Tipo{EjemploDeTAD}
% }
%     \requiere{predicadoUnilinea(caracteres)}
%     \asegura{res.secuenciaDeCaracteres = caracteres}
%     \aseguraLargo{ \norm{res.conjuntoDeTuplas} = \norm{secuenciaDeNombres} \land \\predicadoMultilinea(secuenciaDeNombres, res.conjuntoDeTuplas) }
% \end{proc}

% \predLargo{predicadoMultilinea}{s: \seq{\cha}, conjunto: \conj{\tupla{\seq{\cha}, \seq{\cha}}}}{
%     \paraTodo{i}{\Z}{0 \leq i < \norm{s} 
%     \implicaLuego s[i] \in c}
% }

% \pred{predicadoUnilínea}{s: \seq{\cha}}{
%     \norm{s} > 0
% }

% \aux{unAuxiliar}{k: \Z,l: \Z}{\Z}{\norm{k-l}}

% \end{tad}
